\documentclass{beamer}

\setbeamertemplate{bibliography item}{[\theenumiv]}

\usetheme{Madrid}
%\usecolortheme{seahorse}
\usepackage[brazil]{babel}
\usepackage[utf8x]{inputenc} % acentos diretamente do teclado
\usepackage[T1]{fontenc}
\usepackage{graphicx}


\title[Programação Paralela]{Intel Cilk Plus}
\subtitle{Programação Paralela}
\author{Alexandre Lucchesi%
       \and Jeremias Moreira%
       \and Matheus Braga}
\institute[UnB]{%
    Departamento de Ciência da Computação\\
    Universidade de Brasília, Brasília -- DF\\[1ex]
    \texttt{alexandrelucchesi@gmail.com}\\
    \texttt{jeremias@aluno.unb.br}\\
    \texttt{matheus.mtb7@gmail.com}\\
}
\date[Outubro, 2014]{10 de outubro de 2014}

\begin{document}

\begin{frame}[plain]
    \titlepage%
\end{frame}

\begin{frame}[shrink]{Sumário}
	\tableofcontents
\end{frame}

\section{Introdução}
	\subsection{O que é}
		\begin{frame}{Introdução}
			\begin{itemize}
			    \item Cilk Plus é ``dahora''~\cite{jeffers:2013}.
			\end{itemize}
		\end{frame}
	\subsection{História}
		\begin{frame}
			É da hora mesmo...
		\end{frame}

\section{Principais Funções}
	\subsection{Exemplos}
		\begin{frame}
			\begin{block}{Teste}
				Mais teste
			\end{block}
		\end{frame}

\section{Análise Comparativa}
	\subsection{Gráfico Speedup}
	
\section{Aspectos da Ferramenta}
	\subsection{Vantagens}
	\subsection{Desvantagens}

\section{Conclusão}

\begin{frame}%[allowframebreaks]
\frametitle{Referências Bibliográficas}
    \tiny{\bibliographystyle{abbrv}}
    \bibliography{refs}
\end{frame}

\end{document}

